Computational models in neuroscience typically contain a number of parameters
that are poorly constrained by experimental data. The sources of parameter
uncertainty may be many, including measurement uncertainty and variability in
the parameter values within a population of cells or dynamically within a single
cell. By quantifying the uncertainties in features of the model output we can
determine the effects of measurement uncertainty as well as the uncertainty
arising from the intrinsic variability.

The standard method of accounting for this uncertainty/variability is to assign
a probability distribution instead of a single value to the parameter value, and
then to compute the corresponding uncertainty in the model output with Monte
Carlo methods.  A more recent mathematical framework for estimating
uncertainties is that of Polynomial Chaos expansions (Xiu & Hesthaven, SIAM J.
Sci. Comput., 2005). Polynomial Chaos is much faster than the standard Monte
Carlo methods used for uncertainty quantification as long as the number of
uncertain parameters is relatively low, typically smaller than about twenty.
This is the case for many, if not most, neuroscience models.

Here we present UncertainPy, a novel Python toolbox, tailored to perform
uncertainty quantification in neuroscience models. UncertainPy bases its
uncertainty analysis on quasi-Monte Carlo methods or Polynomial Chaos, depending
on the number of uncertain model parameters. Polynomial Chaos expansions are
obtained from the previously developed package Chaospy (Feinberg & Langtangen,
Journal of Computational Science, 2015). UncertainPy is feature based, i.e., if
applicable, it recognizes and calculates the uncertainty in salient model
response features such as spike timing, action potential shape and similar.
UncertainPy is parallelized, has support for a wide range of different models,
and can easily be customized to new models and features.

To demonstrate UncertainPy, we perform uncertainty analysis of (i) the standard
Hodgkin-Huxley point-neuron model for action-potential generation and (ii) a
comprehensive multi-compartmental model for interneurons in the dorsal lateral
geniculate nucleus (Halnes et al, PLoS Comp Biol, 2011).
