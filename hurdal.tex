%\documentclass[handout]{beamer}
\documentclass[presentation]{beamer}
\usepackage[utf8]{inputenc}
\usepackage{amsmath, pdfpages, pdflscape, lscape, color, listings, hyperref, amssymb, graphicx, textcomp,varioref, afterpage, subcaption, float, bm, tikz, colortbl}

\global
\newcommand{\Fig}[1]{Figure \ref{#1}}
\newcommand{\fig}[1]{figure \ref{#1}}
\newcommand{\tab}[1]{table \ref{#1}}
\newcommand{\eq}[1]{equation \ref{#1}}
\newcommand{\Eq}[1]{Equation \ref{#1}}
\newcommand{\alg}[1]{algorithm \ref{#1}}
\newcommand{\Alg}[1]{Algorithm \ref{#1}}
\newcommand{\chp}[1]{chapter  \ref{#1}}
\newcommand{\Chp}[1]{Chapter  \ref{#1}}
\newcommand{\e}[1]{\cdot 10^{#1}}
\newcommand{\h}{\hbar}
\newcommand{\der}[2]{\frac{\partial #1}{\partial #2}}
\newcommand{\dder}[2]{\frac{\partial^2 #1}{\partial #2^2}}
\newcommand{\p}{\boldsymbol{P}}
\newcommand{\q}{\boldsymbol{q}}
\newcommand{\norm}[1]{\left\lVert#1\right\rVert_{\!Q}}
\newcommand{\inner}[1]{\left\langle#1\right\rangle_{\!Q}}
\newcommand{\coef}[2]{\frac{\inner{#1,#2}}{\norm{#2}^2}}

\newcommand{\gooditem}[1]{\setbeamercolor{item}{fg=green}\item #1}
\newcommand{\baditem}[1]{\setbeamercolor{item}{fg=red}\item #1}

\DeclareMathOperator*{\argmin}{argmin}
\DeclareMathOperator*{\argmax}{argmax}

\newcommand{\E}[1]{\mbox{E}\!\left(#1\right)}
\newcommand{\Var}[1]{\mbox{Var}\!\left(#1\right)}
\newcommand{\Cov}[1]{\mbox{Cov}\!\left(#1\right)}



\setbeamertemplate{blocks}[rounded][shadow=false]


\usefonttheme[onlysmall]{structurebold}
% Use a bold face title font
\setbeamerfont{title}{series=\bfseries}
\setbeamerfont{frametitle}{series=\bfseries}

% This is not created yet, but should be...
%\usecolortheme{simula}

% Suppress navigation symbols
\setbeamertemplate{navigation symbols}{}

\definecolor{listingsstringcolor}{rgb}{0,0.5,0}
\definecolor{listingskeywordcolor}{rgb}{0,0,1}
\definecolor{listingsbasiccolor}{rgb}{0.0,0.0,0.0}

\definecolor{myred}{RGB}{190,0,0}

 \newcommand{\listingsfont}{\sffamily}

 \lstset{
    language=python,
    commentstyle=\color{red},
     basicstyle=\color{listingsbasiccolor},
    keywordstyle=\color{listingskeywordcolor},
    stringstyle=\color{listingsstringcolor},
 tabsize=4,                       % sets default tabsize to 2 spaces
 morekeywords={as, assert, with, yield},
escapeinside={||},
basicstyle=\ttfamily\footnotesize,
columns=fixed
}

\newenvironment{chaospy}[1]
{\color{gray!30!black}
     \usebackgroundtemplate{
   \begin{tikzpicture}[remember picture, overlay]
     \node[anchor = center, opacity=.15] (image) at (current page.center) {\includegraphics[scale=0.25]{chaospy_logo.jpg}};
   \end{tikzpicture}}
     \begin{frame}[fragile, environment=chaospy]
    \frametitle{{#1}}}
{\end{frame}}


\definecolor{keywords}{RGB}{255,0,90}
\definecolor{comments}{RGB}{0,0,113}
\definecolor{red}{RGB}{160,0,0}
\definecolor{green}{RGB}{0,150,0}


\setbeamercolor{title}{fg=myred}
\setbeamercolor{frametitle}{fg=black!65!white,bg=white}
\setbeamercolor{normal text}{fg=black!75!white,bg=white}
\setbeamercolor{structure}{fg=black,bg=white}



\graphicspath{{./figures/}}



\begin{document}

% \maketitle

% \begin{frame}
%    \begin{tikzpicture}[remember picture, overlay]
%      \node[anchor = center, opacity=.25] (image) at (current page.center) {\includegraphics[scale=0.25]{chaospy_logo.jpg}};
%    \end{tikzpicture}
%
% \begin{center}
%     \textbf{ \color{myred} \Large Chaospy: \\ \vspace{1mm} A modular implementation of polynomial\\ \vspace{1mm} chaos expansions and Monte Carlo methods}
% \end{center}
% \begin{center}
%      \large \vspace{5mm} Simen Tennøe\\ \vspace{5mm}\footnotesize Supervisors:\\ \vspace{0.5mm} Jonathan Feinberg, Hans Petter Langtangen, Gaute Einevoll and Geir Halnes
% \end{center}
%
% \begin{center}
%     \small University of Oslo, CINPLA
% \end{center}
% %\includegraphics[width=\textwidth]{chaospy_logo.jpg}
%
% \end{frame}




\begin{frame}
\begin{center}
    \textbf{\color{myred}\Large Uncertainpy: \\
    \vspace{1mm} A Python toolbox for uncertainty\\
    \vspace{1mm} quantification of computational neuroscience \\
    \vspace{1mm}  models}
\end{center}

% \frametitle{\textbf{\color{myred}\Large Chaospy: \\ \vspace{1mm} A modular implementation of polynomial\\ \vspace{1mm} chaos expansions and Monte Carlo methods}}

     \large \vspace{8mm} Simen Tennøe

      \vspace{6mm}
      \footnotesize Supervisors:

    %
      \vspace{1mm}
      Jonathan Feinberg

      Hans Petter Langtangen

      Gaute Einevoll

      Geir Halnes

      \vspace{5mm}
    \small University of Oslo, CINPLA
 %
 % \begin{tikzpicture}[remember picture,overlay]
 %    \node [xshift=0.2\paperwidth,yshift=-0.14\paperheight] at (current page.center)
 %          {\includegraphics[width = 0.5\paperwidth ]{chaospy_logo.jpg}};
 %  \end{tikzpicture}

  \begin{tikzpicture}[remember picture,overlay]
    \node [xshift=1.35cm,yshift=0.55cm] at (current page.south west)
          {\includegraphics[width = 0.2\paperwidth ]{cinpla.png}};
  \end{tikzpicture}

\end{frame}




\begin{frame}[fragile]{Uncertainpy is a Python toolbox for uncertainty quantification of computation neuroscience models}


  %   \begin{tikzpicture}[remember picture, overlay, font=\sffamily]
  %     \node [align=left, xshift=-0.4\textwidth,yshift=0.15\textwidth] (image3) at (current page.center)
  %           {\includegraphics[width = 0.2\textwidth]{chaospy_logo.jpg}};
  %     \node[align=left] at (image3.east) {\hspace{5cm} \bf Properties of Chaospy};
  %   \end{tikzpicture}
  %
  % \pause
%
%   \begin{tikzpicture}[remember picture, overlay, font=\sffamily]
%   \node [align=left, xshift=-0.2\textwidth,yshift=-0.05\textwidth] (image1) at (current page.center)
%       {\includegraphics[width = 0.25\textwidth]{samples.png}};
%   \node[align=left] at (image1.east) {\hspace{4.5cm} \bf Monte Carlo methods};
%   \end{tikzpicture}
%
% \pause
%
%
%     \begin{tikzpicture}[remember picture, overlay, font=\sffamily]
%       \node [align=left, xshift=0\textwidth,yshift=-0.25\textwidth] (image2) at (current page.center)
%             {\includegraphics[width = 0.25\textwidth]{pc.png}};
%       \node[align=left] at (image2.east) {\hspace{4cm} \bf Polynomial Chaos};
%     \end{tikzpicture}

\end{frame}


\begin{frame}{Creating a computational model consists of three steps, model creation, parameter estimation and uncertainty quantification}
    \begin{tikzpicture}[remember picture, overlay, font=\sffamily]
      \onslide<1-3>{\node [align=left, xshift=-0.35\textwidth,yshift=0.15\textwidth] (image3) at (current page.center)
            {$I = C_m\frac{{\mathrm d} V_m}{{\mathrm d} t}  + I_{\text{ion channels}}$};
      \node[align=left] at (image3.east) {\hspace{4cm} \bf Model creation};}
    \end{tikzpicture}
   %
   % \begin{tikzpicture}[remember picture, overlay, font=\sffamily]
   %   \onslide<1-3>{\node [align=left, xshift=-0.4\textwidth,yshift=0.15\textwidth] (image3) at (current page.center)
   %         {\includegraphics[width = 0.4\textwidth]{compartmental.png}};
   %   \node[align=left] at (image3.east) {\hspace{5cm} \bf Model creation};}
   % \end{tikzpicture}

  \begin{tikzpicture}[remember picture, overlay, font=\sffamily]
  \onslide<2-3>{\node [align=left, xshift=-0.2\textwidth,yshift=-0.05\textwidth] (image1) at (current page.center)
      {\includegraphics[width = 0.15\textwidth]{parameters.png}};
  \node[align=left] at (image1.east) {\hspace{4.5cm} \bf Estimate parameters};}
  \end{tikzpicture}

    \begin{tikzpicture}[remember picture, overlay, font=\sffamily]
      \onslide<3-4>{\node [align=left, xshift=0\textwidth,yshift=-0.25\textwidth] (image2) at (current page.center)
            {\includegraphics[width = 0.25\textwidth]{uq.png}};
      \node[align=left] at (image2.east) {\hspace{4cm} \bf Quantify uncertainties};}
    \end{tikzpicture}

\end{frame}



\begin{frame}{Uncertainty quantification is important }

\end{frame}



\begin{frame}{Point wise comparison can be problematic since "the same" spike can occur at different times when varying the parameters}
   \vspace{-5mm}
   \begin{figure}
      \only<1>{\includegraphics[width=0.8\textwidth]{hh_thin_extract.png}}
      \only<2>{\includegraphics[width=0.8\textwidth]{hh_thin_extract_elipse.png}}
   \end{figure}
\end{frame}



\begin{frame}{90\% confidence interval for the Hodgkin-Huxley model with original uncertainties in the parameters}
   \vspace{-5mm}
\begin{figure}
   \only<1>{\includegraphics[width=0.8\textwidth]{confidence-interval.png}}
   \only<2>{\includegraphics[width=0.8\textwidth]{confidence-interval_elipse.png}}
\end{figure}
\end{frame}


\begin{frame}{The solution is to calculate the uncertainty for features such as the number of spikes}
\vspace{-5mm}
\begin{figure}
   \includegraphics[width=0.8\textwidth]{hh_thin_extract_nrSpikes.png}
\end{figure}
\end{frame}


\begin{frame}{Sensitivity of the result for each parameter in the Hodgkin-Huxley model}
\vspace{-5mm}
\begin{figure}
   \includegraphics[width=1\textwidth]{sensitivity_grid.png}
\end{figure}
\end{frame}


  \begin{frame}{Summary: UncertainPy is a novel Python toolbox, tailored to perform uncertainty quantification in neuroscience models}

  \begin{tikzpicture}[remember picture, overlay, font=\sffamily]
      \node [align=left, xshift=0.49\textwidth,yshift=-0.37\textwidth] (image2) at (current page.center)
            {\includegraphics[width = 0.35\textwidth]{cinpla.png}};
    \end{tikzpicture}

   %  \begin{tikzpicture}[remember picture, overlay, font=\sffamily]
   %      \node [align=left, xshift=-0.45\textwidth,yshift=-0.37\textwidth] (image2) at (current page.center)
   %            {\includegraphics[width = 0.18\textwidth]{chaospy_logo.jpg}};
   %    \end{tikzpicture}

\pause
\begin{tikzpicture}[remember picture, overlay, font=\sffamily]

  \node[align=left, yshift=0.15\textwidth] at (current page.south){ \bf \large Questions?};
\end{tikzpicture}



\end{frame}


\end{document}
